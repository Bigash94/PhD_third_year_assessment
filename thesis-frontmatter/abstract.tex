\chapter{Abstract}

To understand the origin of the solar system's elemental abundances is a major challenge in astrophysics. Many generations of past stars and processes have contributed to the solar abundances. The abundances of the heavy elements beyond iron that we observe today in the solar system are mainly the result of two nucleosynthesis processes: the \acrfull{spro} and the \acrfull{rpro} . Low-mass \acrfull{agb} (2 $<$ $M$/$M_\odot$ $<$ 3) and massive ($M$/$M_\odot$ $>$ 10) stars have been identified as the sites of the $s$-process. With the calculation of a new set of low-mass \acrshort{agb} models with initial masses $M$/$M_\odot$ $=$ 2,3 and Z = 0.01, 0.02 and 0.03 improvements to their element production is aided by enhanced mixing. The $^{13}$C-pocket, a thin shell in the star where \acrshort{spro} material is produced, is formed through mixing enhanced by \acrfull{igw} interactions. This results in a pocket three times larger on average and hence larger \acrshort{spro} production compared to previous models using overshoot, the use of momentum within convective elements overshooting the boundary between the hydrogen envelope and helium shell, inducing mixing between these layers. The full nucleosynthesis was calculated in post-processing using the NuGrid mppnp code. Isotopic and elemental abundances are compared to other stellar data sets available in the literature and to a wide range of observations, including carbon-stars, barium stars, post-\acrshort{agb} stars, and pre-solar grains. Good agreement was determined with few exceptions, for certain molybdenum an barium isotopes. To judge the significance of (dis)agreements of our model results with observations, I determined uncertainties of stellar yields due to uncertainties in nuclear reaction rates by adopting a Monte Carlo method. With this method the reaction rates can be simultaneously varied within their uncertainties to produce changes in abundances of isotopes. These variations in abundances and reaction rates are correlated. The strength of the correlation defines the importance of the reaction to influence certain isotopic abundances. 