\chapter{Abstract}

To understand the origin of the solar system elemental abundances is a major challenge in astrophysics. The abundances of the heavy elements beyond iron that we observe today in the solar system are mainly the result of the two nucleosynthesis processes: the slow neutron capture ($s$-) process and the rapid neutron capture ($r$-) process. 
%Many generations of past stars and processes have contributed to the solar abundances. 
Low-mass \acrfull{agb} (2 $<$ $M$/$M_\odot$ $<$ 3) and massive ($M$/$M_\odot$ $>$ 10) stars have been identified as the sites of the $s$-process. We provide a new set of low-mass \acrshort{agb} models with initial masses $M$/$M_\odot$ $=$ 2,3 and Z = 0.01, 0.02 and 0.03. 
%Internal gravity wave mixing is the physics mechanism responsible for the formation of a $^{13}$C-pocket on average three times larger than our previous data set. Consequently the acrshort{spro} production is significantly enhanced within the range of the increased $^{13}$C-pocket size. 
The 13C pocket, a thin shell in the star where the bulk of s-process is being produced, is formed by gravity wave mixing . This results in an on average three times larger pocket and hence s-process efficiency compared to previous models using overshoot.



Abundances are compared to other stellar data sets available in the literature and to a wide range of observations, including carbon-stars, barium stars, post-\acrshort{agb} stars, and pre-solar grains. The full nucleosynthesis was calculated in post-processing using the NuGrid mppnp code.

Included is the validation of a grid of new stellar models that simulate stars classified as low-mass \acrshort{agb} from the \acrlong{pms} phase through to the end of the \acrshort{agb} phase. This grid consists of the six models mentioned above. Calculated from these stellar models are the evolution of isotopic abundances which were compared to physically measured abundances from stars and grains. Inspecting the nucleosynthesis outputs produces the stellar yields of each isotope in the nuclear network, and are produced for use in \acrfull{gce} codes.



1) A new set of AGB models was used to calculate nucleosynthesis
2) in these models, the 13C pocket, which is...., is formed by gravity wave mixing
3) this results in a higher s process efficiency due to a larger 13C pocket, in contrast to previous models where then pocket is formed via....
4) we have investigaed a grid of models of M=... and Z=...
5) the fulls nucleosynthesis was calcualted with the  post processing code mppnp
6) The evolution of isotopic abundances was calculated and compared to observations od.... + what is the result? where do you agree, where not? how is that differentfrom previous models?
7) To investigate if disagreements are due to stellar models, or uncertainties in the nuclear reaction rates, uncertainties of the stellar yields due to nuclear physics uncertainties were determined
8) this was achieved by Monte Carlo techniques.... (explain how you do it)
9) and then what your result was of this study, but obviously we dont know that yet.  

I would avoid "we have": this is about you,so it is easie



Additionally, To better understand why theory and observation are not in good agreement, investigation into the uncertainty of these yields are required. Sources of uncertainty lie in the models themselves as well as the reaction rates that determine the abundance of the stable isotopes. In addition to these yields, the uncertainty in mass fraction of each stable isotope heavier than iron will be estimated for all models described above.  
