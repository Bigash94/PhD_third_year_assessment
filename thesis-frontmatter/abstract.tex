\chapter{Abstract}

To understand the origin of the solar system elemental abundances is a challenge being worked towards internationally. The abundances of the heavy elements beyond iron that we observe today in the solar system are mainly the result of the two nucleosynthesis processes: the slow neutron capture ($s$-) process and the rapid neutron capture ($r$-) process. Many generations of past stars and processes have contributed to the solar abundances. Low-mass \acrfull{agb} (2 $<$ $M$/$M_\odot$ $<$ 3) and massive ($M$/$M_\odot$ $>$ 10) stars have been identified as the sites of the $s$-process. We provide a new set of low-mass \acrshort{agb} models with initial masses $M$/$M_\odot$ $=$ 2,3 and Z = 0.01, 0.02 and 0.03. Internal gravity wave mixing is the physics mechanism responsible for the formation of a $^{13}$C-pocket on average three times larger than our previous data set. Consequently the acrshort{spro} production is significantly enhanced within the range of the increased $^{13}$C-pocket size. Abundances are compared to other stellar data sets available in the literature and to a wide range of observations, including carbon-stars, barium stars, post-\acrshort{agb} stars, and pre-solar grains. The full nucleosynthesis was calculated in post-processing using the NuGrid mppnp code.

This thesis includes the validation of a grid of new stellar models that simulate stars classified as low-mass \acrshort{agb} from the \acrlong{pms} phase through to the end of the \acrshort{agb} phase. This grid consists of the six models mentioned above. Calculated from these stellar models are the evolution of isotopic abundances which were compared to physically measured abundances from stars and grains. Inspecting the nucleosynthesis outputs produces the stellar yields of each isotope in the nuclear network, and are produced for use in \acrfull{gce} codes. 

Additionally, To better understand why theory and observation are not in total agreement, investigation into the uncertainty of these yields are required. Sources of uncertainty lie in the models themselves as well as the reaction rates that determine the abundance of the stable isotopes. In addition to these yields, the uncertainty in mass fraction of each stable isotope heavier than iron will be estimated for all models described above.  
