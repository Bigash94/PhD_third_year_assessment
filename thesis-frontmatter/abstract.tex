\chapter{Abstract}

To understand the origin of the solar system's elemental abundances is a major challenge in astrophysics. Many generations of past stars and processes have contributed to the solar abundances. The abundances of the heavy elements beyond iron that we observe today in the solar system are mainly the result of two nucleosynthesis processes: the \acrfull{spro} and the \acrfull{rpro} . Low-mass \acrfull{agb} (2 $<$ $M$/$M_\odot$ $<$ 3) and massive ($M$/$M_\odot$ $>$ 10) stars have been identified as the sites of the $s$-process. Introducing a new set of low-mass \acrshort{agb} models with initial masses $M$/$M_\odot$ $=$ 2,3 and Z = 0.01, 0.02 and 0.03. The $^{13}$C-pocket, a thin shell in the star where \acrshort{spro} material is produced, is formed through mixing enhanced by \acrfull{igw} interactions. This results in a pocket three times larger on average and hence larger \acrshort{spro} production compared to previous models using overshoot. The full nucleosynthesis was calculated in post-processing using the NuGrid mppnp code. Isotopic and elemental abundances are compared to other stellar data sets available in the literature and to a wide range of observations, including carbon-stars, barium stars, post-\acrshort{agb} stars, and pre-solar grains. Good agreement was determined with few exceptions, for certain molybdenum an barium isotopes. I investigated if disagreements where related to model or nuclear uncertainties, along with the calculation of uncertainties in the stellar yields. They were achieved by adopting a Monte Carlo method to correlate reaction rates with isotopic abundances.