\chapter{Stellar evolution and nucleosynthesis of low-mass stars}

Completion = 20\% 
A rough detailed outline has been made further detailing is required to complete.

\section{The stellar evolution of different mass ranges}

In this section I will describe what the difference in categories of stars derived by mass listed below.

\subsection{The lowest mass stars}

\subsection{The intermediate-mass stars}

\subsubsection{The lower intermediate mass stars}

\subsubsection{The middle intermediate mass stars}

\subsubsection{The massive intermediate mass stars}

\subsection{The massive stars}

\section{Evolution of low-mass AGB stars}

Describes the evolution of low-mass stars that MESA describes, WD is the exception as it is calculated but I did not study it.

\subsection{Pre-main sequence (PMS)}

\subsection{Main sequence (MS)}

\subsection{Red giant branch (RGB)}

\subsection{Horizontal branch (HB)}

\subsection{Asymptotic giant branch (AGB)}

Outlines the heavy element production stage of low-mass stars.

\subsection{White dwarf (WD)}

\section{Neutron capture nucleosynthesis}

Answers the question of how heavy elements are made in stars.

\subsection{The slow neutron capture process}

This is the production method looked at in this thesis. Details of what this is and why it happens.

\subsection{The intermediate neutron capture process}

Details the possibility of i-process, potential sites as AGB can have i-process. The models exhibit this behaviour but it can not be determined without extra investigation.

\subsection{The rapid neutron capture process}

Informs the reader what the r-process entails.